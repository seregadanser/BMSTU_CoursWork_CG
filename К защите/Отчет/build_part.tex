\chapter{Конструкторская часть}
\section{Требования к разрабатываемой программе}
Программа должна предоставлять следующий доступ к функционалу:
\begin{itemize}
	\item перемещения, маштабирования, поворота ребер, вершин, полигонов модели и самой модели целиком;
	\item добавления, удаления вершин, ребер и полигонов;
	\item изменения координат вершин;
	\item выбора различных алгоритмов вывода модели на экран;
	\item загрузки и сохранению модели в файл;
	\item свободному перемещению по сцене без перемещения модели;
	\item возможности прекратить и возобновить отрисовку в любой момент.
\end{itemize}
Требования к программе:
\begin{itemize}
	\item программа должна корректно обрабатывать действия пользователя;
	\item программа должна реагировать на действия пользователя с задержкой не более секунды, либо выводить соответствующее сообщение об обработке запроса.
\end{itemize}
\section{ Выбор используемых типов и структур данных}
Для разрабатываемо программного обеспечения необходимо реализовать следующие типы:
\begin{itemize}
	\item математические абстракции --- точки, линии, вектора, матрицы;
	\item модель, задаваемая вершинами, полигонами, ребрами;
	\item камера --- наблюдатель;
	\item интерфейс --- позволяет пользователю взаимодействовать со сценой;
	\item сцена --- хранит камеру и модель, реализует обработку отрисовки;
	\item посетители --- для унифицирования разработки программы и упрошения дальнейшего развития и дополнения.
\end{itemize}
Нужно разработать следующие структуры данных: 
\begin{itemize}
	\item хэш таблицы --- для хранения элементов модели;
	\item списки активных элементов;
	\item контейнерный класс --- для хранения связей между элементами модели.
\end{itemize}
\section{Описание алгоритмов}
На рисунках \ref{ris:imageS} и \ref{ris:imageS1} представлены схемы двух алгоритмов отсечения невидимых ребер --- z-буфера и рэйкастинга соответственно. 
\begin{center}
\centering
\def\svgwidth{7cm}
\input{src/zbuf.pdf_tex}
\captionof{figure}{Схема алгоритма z-буфера}
\label{ris:imageS}

\newpage

\def\svgwidth{15cm}
\input{src/raycasting.pdf_tex}
\captionof{figure}{Схема алгоритма рэйкастинга}
\label{ris:imageS1}
\end{center}

\section{Описание способа хранения элементов}
Для корректного взаимодействия элементов необходимо хранить связи между объектами. Для этого создана структура контейнера, котрая выполняет данную работу. Связь между объектами аналогична связи между детьми и родителями так как для каждого из трех типов элементов легко выделить такие отношения: 
\begin{itemize}
	\item для полигона детьми являются три ребра и три вершины;
	\item у ребра есть родители --- полигоны --- и потомоки --- две вершины;
	\item для вершины родителями являются полигоны и ребра.
\end{itemize}  
Контейнер включает в себя три хэш таблицы --- для хранения элементов, для хранения родителей элемента, для хранения детей элемента. Если у элемента нет родителей или детей, соответствующие таблицы не заполняются. 

Удаление частей модели из контейнера происходит по двум принципам:
\begin{enumerate}
	\item[1)] если у элемента не осталось ни одного родителя, то он подлежит удалению;
	\item[2)] если у элемента пропал хоть один потомок, то он так же подлежит удалению. 
\end{enumerate}  
При добавлении элемента в контейнер добавляется сам элемент и индетификаторы связанных объектов.

Отдельно стоит отметить, что в контейнере не может храниться одновременно несколько абсолютно одинаковых объектов. 