%\chapter*{\centering{Введение}}
%\addcontentsline{toc}{chapter}{Введение}
\ssr{Введение}

Целью курсовой работы является разработка универсального программного обеспечения, которе позволит: 
\begin{enumerate}
	\item[1)] загружать параметры трехмерной модель из файла;
	\item[2)] редактировать трехмерные модели на уровне вершин, ребер, полигонов; 
	\item[3)] просматривать трехмерную модель как с отсечением невидимых ребер и граней, так и без отсечения с любой позиции наблюдателя.
\end{enumerate}

Для достижения поставленной цели требуется решить ряд задач:
\begin{enumerate}
	\item[1)] провести анализ алгоритмов для удаления невидимых линий, закраски, освещения моделей и выделить наиболее подходящие для будущей программы;
	\item[2)] реализовать выбранные алгоритмы и структуры данных;
	\item[3)] разработать программное обеспечение для отображения трехмерной
	сцены;
	\item[4)] выполнить исследование на основе разработанной программы.
\end{enumerate}

\newpage
